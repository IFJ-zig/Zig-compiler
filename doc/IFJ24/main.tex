\documentclass[a4paper, 12pt]{article}
\usepackage[utf8]{inputenc}
\usepackage[czech]{babel}
\usepackage[left=2cm, top=3cm, text={17cm, 24cm}]{geometry}
\usepackage{graphicx}
\usepackage{fancyhdr}
\usepackage{enumitem}
\usepackage[T1]{fontenc}
\usepackage{tikz}
\usepackage{tikz-qtree}
\usetikzlibrary{shapes.geometric}
\usetikzlibrary{calc,shapes.multipart,chains,arrows}
\makeatletter
\newcommand*{\rom}[1]{\expandafter\@slowromancap\romannumeral #1@}
\makeatother
\usepackage[unicode]{hyperref}
\hypersetup{
	colorlinks = true,
	hypertexnames = true,
	citecolor = red
}

\setlength{\headheight}{15pt}

\begin{document}

    %Titulni strana
    \begin{titlepage}
        \begin{center}
            \includegraphics[width=0.87\textwidth]{images/logo_cz.png}
            \vspace*{6cm}

            \Huge{\textbf{Dokumentace}}
            \vspace{0.5cm}
            
            \LARGE{Implementace překladače imperativního jazyka IFJ24}
            \vspace{0.5cm}
            
            \Large{Tým xbohatd00, Varianta - TRP-izp}
            \vspace{2.5cm}
            
            \large{\textbf{Daniel Bohata} (xbohatd00) 25\%}
            \vspace{0.1cm}
            
            \large{Tadeáš Horák (xhorakt00) 25\%}
            \vspace{0.1cm}
            
            \large{Ivo Puchnar (xpuchn02) 25\%}
            \vspace{0.1cm}
            
            \large{Adam Vožda (xvozdaa00) 25\%}
            \vspace{0.1cm}
            
           \vfill
		   \begin{flushleft} 
		   \large
		   Bez rozšíření
		   \hfill
		   Brno, \today
		   \end{flushleft}
        \end{center}
    \end{titlepage}

\pagestyle{fancy}
\lhead{\bfseries Překladač IFJ24}
\rhead{\bfseries Obsah}

\newpage
\tableofcontents


\newpage
\rhead{\bfseries Práce v týmu}
\section{Práce v týmu}
\subsection{Rozdělení práce mezi jednotlivé členy týmu}
\begin{itemize}
    \begin{minipage}{0.25\linewidth}   
        \item Daniel Bohata
        \item[]\item[]\item[]
        \item Tadeáš Horák
        \item[]\item[]\item[]
        \item Ivo Puchnar
        \item[]\item[]\item[]
        \item Adam Vožda
        \item[]\item[]\item[]
        \vspace{0.25cm}
    \end{minipage}
    \begin{minipage}{0.7\linewidth}
        \item[-] návrh
        \item[-] datový
        \item[-] obousměrně
        \item[]
        
        \item[-] návrh
        \item[-] implementace
        \item[-] zásobník
        \item[]
        
        \item[-] návrh
        \item[-] datové
        \item[-] návrh
        \item[]
        
        \item[-] návrh
        \item[-] token\_t
        \item[-] implementace 
    \end{minipage}
\end{itemize}

%%%%%%%%%%%%%%%%%%%%%%%%%%%%%%%%%%%%%%%%%%%%%%%%%%%%%%%%%%%%%%%%%%%%%
\newpage
\rhead{\bfseries Implementace překladače (Lexikální analýza)}
\section{Implementace překladače}
\subsection{Lexikální analýza}
%\subsubsection{Základní struktura lexikálního analyzátoru}
Lexikální analýza se provádí funkcí \textit{get\_token()} 
funkce \textit{\textbf{get\_next\_token()}} Rozhraní (\textit{scanner}) načítající \newline
\textit{\textbf{delete\_token()}} která

\subsubsection{Vysvětlení konečného automatu}
\begin{itemize}
    \item Ovály s jednoduchým okrajem jsou obyčejné stavy a ovály s dvojitým okrajem jsou stavy konečné.
    \item Pro přehlednost jsou některé přechody mezi stavy spojeny do jedné čáry. Pokud tomu tak je, symbol, který přechodu náleží, je vyobrazen tak, aby se nedaly zaměnit. Buď u jednotlivých přechodů, nebo u přechodů se stejným znakem směřujících do stejného stavu. 
    \item Některé přechody mají šipku oběma směry. To znamená, že se při správném znaku lze vrátit do předchozího stavu.
    \item Pokud automat dojde do konečného stavu a dostane znak neodpovídající žádnému z jeho přechodů, znak vrátí na vstup a odešle token odpovídající aktuálnímu konečnému stavu s případným obsahem (číslo/string).
    \item Pokud automat dojde do běžného stavu a dostane znak neodpovídající žádnému z jeho přechodů, dochází k chybě číslo 1, chyba v lexému, a program končí.
    \item Automat začíná velkým stavem start vlevo uprostřed. Vlevo nahoře se nachází oblast tokenů tvořených 1-2 znaky. Vpravo nahoře jsou klíčová slova a identifikároty proměnných a funkcí. Vpravo uprostřed jsou víceřádkové stringy. Vpravo dole jsou jednořádkové stringy
\end{itemize}

\newpage
\subsubsection{Diagram konečného automatu}
\begin{figure}[ht!]
    \begin{center}
        \includegraphics[origin=c, width=0.8\textwidth]{images/FSM_IFJ24.drawio.png}
        \caption{Diagram konečného automatu}
    \end{center}
\end{figure}

%%%%%%%%%%%%%%%%%%%%%%%%%%%%%%%%%%%%%%%%%%%%%%%%%%%%%%%%%%%%%%%%
\newpage
\rhead{\bfseries Implementace překladače (Syntaktická analýza)}
\subsection{Syntaktická analýza}
\subsubsection{Parser}
Syntaktická  \cite{FITPUB8538}. Načtení \textit{\textbf{next\_token()}}. uvolnění \textit{get\_next\_token()} načtení 

\subsubsection{Rozhraní mezi parserem a zpracováním výrazů (psa)}
Rozhraní \textit{\textbf{psa()}}. zapouzdřena \textit{expression()}. Jako parametr

\vspace{4cm}

\begin{figure}[ht!]
\begin{center}
\begin{tikzpicture}[square/.style={regular polygon,regular polygon sides=4}]
        \node at (0,0) [square, draw] {PARSER};
        \node at (12,0) [square, draw] {Zpracování výrazů};
        \draw (1.4,1) -- node[above]{psa()} (9.45,1);
        \draw (1.4,-1) -- node[above]{next\_token()} (9.45,-1);
\end{tikzpicture}
\caption{Rozhraní mezi parserem a zpracováním výrazů}
\end{center}
\end{figure}

\newpage

\subsubsection{LL gramatika}

\begin{enumerate}
    \item <prog> → require “ifj21” <main\_b>
    \item[]
    \item <main\_b> → function id (<params>) <ret\_func\_types> <stats> end <main\_b>
    \item <main\_b> → global id : function (<arg\_def\_types>) <ret\_def\_types> <main\_b>
    \item <main\_b> → id (<args>) <main\_b>
    \item <main\_b> → $\varepsilon$
    \item[]
    \item <stats> → local id : <type> <assign> <stats>
    \item <stats> → if exp then <stats> else <stats> end <stats>
    \item <stats> → while exp do <stats> end <stats>
    \item <stats> → return <ret\_vals> <stats>
    \item <stats> → id <id\_func> <stats>
    \item <stats> → $\varepsilon$
    \item[]
    \item <id\_func> → <n\_ids> = <as\_vals>
    \item <id\_func> → (<args>)
    \item[]
    \item <params> → id : <type> <n\_params>
    \item <params> → $\varepsilon$
    \item <n\_params> → , id : <type> <n\_params>
    \item <n\_params> → $\varepsilon$
    \item[]
    \item <n\_ids> → , id <n\_ids>
    \item <n\_ids> → $\varepsilon$
    \item[]
    \item <vals> → exp <n\_vals>
    \item <n\_vals> → , exp <n\_vals>
    \item <n\_vals> → $\varepsilon$
    \item[]
    \item <as\_vals> → <vals>
    \item <as\_vals> → id (<args>)
    \item[]
    \item <ret\_vals> → <vals>
    \item <ret\_vals> → $\varepsilon$
    \item[]
    \item <assign> → = <assign\_val>
    \item <assign> → $\varepsilon$
    \item[]
    \item <assign\_val> → exp
    \item <assign\_val> → id (<args>)
    \item[]
    \item <term> → id
    \item <term> → <const>
    \item[]
    \item <args> → <term> <n\_args>
    \item[]
    \item <args> → $\varepsilon$
    \item <n\_args> → , <term> <n\_args>
    \item <n\_args> → $\varepsilon$
    \item[]
    \item <arg\_def\_types> → <func\_def\_types>
    \item <arg\_def\_types> → $\varepsilon$
    \item[]
    \item <ret\_func\_types> → : <func\_types>
    \item <ret\_func\_types> → $\varepsilon$
    \item[]
    \item <ret\_def\_types> → : <func\_def\_types>
    \item <ret\_def\_types> → $\varepsilon$
    \item[]
    \item <func\_types> → <type>  <n\_func\_types>
    \item <n\_func\_types> → , <type>  <n\_func\_types>
    \item <n\_func\_types> → $\varepsilon$
    \item[]
    \item <func\_def\_types> → <type>  <n\_func\_def\_types>
    \item <n\_func\_def\_types> → , <type>  <n\_func\_def\_types>
    \item <n\_func\_def\_types> → $\varepsilon$
    \item[]
    \item <type> → integer
    \item <type> → number
    \item <type> → string
    \item <type> → nil
    \item[]
    \item <const> → int\_value
    \item <const> → double\_value
    \item <const> → string\_value
    \item <const> → nil
\end{enumerate}

\vspace{1cm}
Poznámka: "exp" - označení pro výraz

\newpage

\subsubsection{LL tabulka}
\begin{figure}[ht!]
\begin{center}
  \includegraphics[angle=90,origin=c, width=0.59\textwidth, trim={0 2.5cm 0 2.5cm},clip]{images/LL_table_1.pdf}
  \caption{LL tabulka, část 1}
\end{center}
\end{figure}


\newpage

\begin{figure}[ht!]
\begin{center}
  \includegraphics[angle=90,origin=c, width=0.59\textwidth, trim={0 2.5cm 0 2.5cm},clip]{images/LL_table_2.pdf}
  \caption{LL tabulka, část 2}
\end{center}
\end{figure}

\newpage

\subsubsection{Zpracování výrazů}
Zpracování \textit{psa.c (.h)} prováděno \cite{FITPUB8538}. Nejprve \textit{psa\_table\_symbol\_enum} načten \textit{\textbf{get\_index\_enum()}} funkce

\subsubsection{Precedenční tabulka}
\begin{figure}[ht!]
\begin{center}
  \includegraphics[width=1\textwidth, trim={0 2.5cm 0 0},clip]{images/precedence_table.pdf}
  \caption{Precedenční tabulka}
\end{center}
\end{figure}

\newpage

\subsubsection{Gramatika pro výrazy}

\begin{enumerate}
    \item E → i
    \item E → $\#$E
    \item E → (E)
    \item E → E $..$ E
    \item E → E $+$ E
    \item E → E $-$ E
    \item E → E $*$ E
    \item E → E $/$ E
    \item E → E $//$ E
    \item E → E $=$ E
    \item E → E $\sim=$ E
    \item E → E $<=$ E
    \item E → E $>=$ E
    \item E → E $<$ E
    \item E → E $>$ E
\end{enumerate}

%%%%%%%%%%%%%%%%%%%%%%%%%%%%%%%%%%%%%%%%%%%%%%%%%%%%%%%%%%%%%%%%%%%%%
\newpage
\rhead{\bfseries Implementace překladače (Tabulka symbolů)}
\subsection{Tabulka symbolů}
Dle výběru varianty zadání \rom{1} je tabulka symbolů implementována binárním vyhledávacím stromem. Ten je implementován v souboru \textit{symtable.c (.h)}.

\subsubsection{Návrh tabulky symbolů}
Prvek v tabulce symbolů obsahuje následující elementy:

\vspace{1cm}

\begin{itemize}
    \begin{minipage}{0.3\linewidth}   
    \item declared
    \item defined
    \item data\_type
    \item params\_count
    \item params\_type\_count
    \item returns\_def\_count
    \item returns\_count    
    \item first\_param
    \item first\_type\_param
    \item first\_def\_ret
    \item first\_ret
    \end{minipage}
    \begin{minipage}{0.65\linewidth}   
    \item[-] proměnná/funkce byla deklarována
    \item[-] proměnná/funkce byla definována
    \item[-] datový typ proměnné
    \item[-] počet parametrů definice funkce
    \item[-] počet parametrů deklarace funkce
    \item[-] počet návratových typů definice funkce
    \item[-] počet návratových typů deklarace funkce    
    \item[-] ukazatel na první parametr definice funkce
    \item[-] ukazatel na první parametr deklarace funkce
    \item[-] ukazatel na první návratový typ definice funkce
    \item[-] ukazatel na první návratový typ deklarace funkce
    \end{minipage}
\end{itemize}

\vspace{1cm}

Všechny \textit{symData\_t}. Při inicializaci

Jako element

\newpage

\subsubsection{Uložení tabulek symbolů do jednosměrně vázaného seznamu}
Jednotlivé \textit{sym\_linked\_list.c (.h)}. Tento

\vspace{4cm}

\begin{figure}[ht!]
\begin{center}
\begin{tikzpicture}[list/.style={rectangle split, rectangle split parts=2,
    draw, rectangle split horizontal}, >=stealth, start chain]

  \node[list,on chain] (A) {lok. tab. symb. *};
  \node[list,on chain] (B) {lok. tab. symb. *};
  \node[list,on chain] (C) {glob. tab. symb. *};
  \node[on chain,draw,inner sep=6pt] (D) {};
  \draw (D.north east) -- (D.south west);
  \draw (D.north west) -- (D.south east);
  \draw[*->] let \p1 = (A.two), \p2 = (A.center) in (\x1,\y2) -- (B);
  \draw[*->] let \p1 = (B.two), \p2 = (B.center) in (\x1,\y2) -- (C);
  \draw[*->] let \p1 = (C.two), \p2 = (C.center) in (\x1,\y2) -- (D);
\end{tikzpicture}

\tikzset{every tree node/.style={minimum width=2em,draw,circle},
         blank/.style={draw=none},
         edge from parent/.style=
         {draw,edge from parent path={(\tikzparentnode) -- (\tikzchildnode)}},
         level distance=1.5cm}
\begin{tikzpicture}[sibling distance=25pt]
\Tree
[.ida     
    [.idb ]
    [.idc 
    \edge[blank]; \node[blank]{};
    \edge[]; [.idd
         ]
    ]
]
\end{tikzpicture}
\begin{tikzpicture}[sibling distance=25pt]
\Tree
[.ide     
    [.idf ]
    [.idg 
    \edge[blank]; \node[blank]{};
    \edge[]; [.idh
         ]
    ]
]
\end{tikzpicture}
\begin{tikzpicture}[sibling distance=30pt]
\Tree
[.i()      
    [.j()  ]
    [.k()  
    \edge[blank]; \node[blank]{};
    \edge[]; [.l()
         ]
    ]
]
\end{tikzpicture}
\vspace{2cm}

* - ukazatel na kořen binárního stromu reprezentujícího tabulku symbolů \newline

\caption{Diagram jednosměrně vázaného seznamu}
\end{center}
\end{figure}

%%%%%%%%%%%%%%%%%%%%%%%%%%%%%%%%%%%%%%%%%%%%%%%%%%%%%%%%%%%%%%%%%%%%%%%%%%%%%%%
\newpage
\rhead{\bfseries Implementace překladače (Sémantická analýza)}
\subsection{Sémantická analýza}
\subsubsection{Parser}
Sémantický \textit{parser.c} Zmíněný

\subsubsection{Zpracování výrazů}
Psa provádí \textit{\textbf{get\_type()}} Při redukci 

\subsection{Generování kódu}
\subsubsection{Vlastní generování kódu}
případů \textit{scale} přidávána

\subsubsection{Rozhraní generátoru kódu}
rozhraní \textit{codeGen\_ ... ()} a ve tvaru \textit{generate\_ ... ()}. Každá v souboru \textit{code\_generator.c (.h)}.

\vspace{0.75cm}

\begin{figure}[ht!]
\begin{center}
\begin{tikzpicture}[square/.style={regular polygon,regular polygon sides=4}]
        \node at (0,0) [square, draw] {PARSER, PSA};
        \node at (12,0) [square, draw] {Generátor kódu};
        \draw [line width=0.75mm] (2.15,0) -- node[above]{codeGen\_ ... (), generate \_ ... ()} (9.8,0);
\end{tikzpicture}
\caption{Rozhraní mezi parserem, psa a generátorem kódu}
\end{center}
\end{figure}

%%%%%%%%%%%%%%%%%%%%%%%%%%%%%%%%%%%%%%%%%%%%%%%%%%%%%%%%%%%%%%%%%%%%%%%%%%%%%%%%%%%%%
\newpage
\rhead{\bfseries Datové struktury a datové typy}
\section{Datové struktury a datové typy}
\subsection{Datové struktury}

\subsubsection{Zásobník symbolů (zpracování výrazů)}
Zásobník \textit{symstack.c (.h)}. Zmíněný \textit{\textbf{symbol\_stack\_insert\_after\_top\_terminal()}}. Funkce

\subsubsection{Zásobník parametrů (parser)}
Zásobník \textit{paramstack.c (.h)}

\subsubsection{Binární vyhledávací strom (tabulka symbolů)}
Binární \textit{symtable.c (.h)}. Jeho \cite{pruvodce2017}. Při 

\subsubsection{Jednosměrně vázaný seznam (tabulka symbolů)}
Jednosměrně \textit{sym\_linked\_list.c (.h)}. Jako

\subsubsection{Jednosměrně vázaný seznam (identifikátory)}
Jednosměrně \textit{ids\_list.c (.h)}. Jeho 

\subsubsection{Obousměrně vázaný seznam (generování kódu)}
Obousměrně \textit{dll.c (.h)}. Slouží \textit{\textbf{DLL\_InsertLast()}}, pomocí \textit{\textbf{DLL\_PrintAll()}}

\subsection{Datové typy}
\subsubsection{string\_t}
Datový \textit{string\_t} implementován

\subsubsection{token\_t}
Datový \textit{token\_t} implementován

\newpage

\rhead{\bfseries Reference}
\bibliographystyle{plain}
\bibliography{reference.bib}

\end{document}
